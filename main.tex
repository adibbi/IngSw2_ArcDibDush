\documentclass[a4paper]{article}

%% Language and font encodings
\usepackage[english]{babel}
\usepackage[utf8x]{inputenc}
\usepackage[T1]{fontenc}


% From https://gist.github.com/timvdalen/3796300
% From https://hackage.haskell.org/package/alloy-1.2.0/src/tutorial/tutorial.tex
\usepackage{color}
\usepackage{listings} % For Alloy code highlights
% alloy.sty
% Alloy mode for the LaTeX listings package.
% This is public domain

\lstdefinelanguage{alloy}{
  keywords={%
      assert, pred, all, no, lone, one, some, check, run,
      but, let, implies, not, iff, in, and, or, set, sig, Int, int,
      if, then, else, exactly, disj, fact, fun, module, abstract,
      extends, open, none, univ, iden, seq,
  },
  literate=%
    {:}{$\colon$}1
    {|}{$\bullet$}1
    {==}{$=$}1
    {=}{$=$}1
    {!=}{$\neq$}1
    {&&}{$\land$}1
    {||}{$\lor$}1
    {<=}{$\le$}1
    {>=}{$\ge$}1
    {all}{$\forall$}1
    {exists}{$\exists$}1
    {!in}{$\not\in$}1
    {\\in}{$\in$}1
    {=>}{$\implies$}2
    % the following isn't actually Alloy, but it gives the option to produce nicer latex
    {|=>}{$\Rightarrow$}2
    {<=set}{$\subseteq$}1
    {+set}{$\cup$}1
    {*set}{$\cap$}1
    {==>}{$\Longrightarrow$}3
    {<==>}{$\Longleftrightarrow$}4
    {...}{$\ldots$}1
    {\\hl}{$\hline$}1
    {\\alpha}{$\alpha$}1
    {\\beta}{$\beta$}1
    {\\gamma}{$\gamma$}1
    {\\delta}{$\delta$}1
    {\\epsilon}{$\epsilon$}1
    {\\zeta}{$\zeta$}1
    {\\eta}{$\eta$}1
    {\\theta}{$\theta$}1
    {\\iota}{$\iota$}1
    {\\kappa}{$\kappa$}1
    {\\lambda}{$\lambda$}1
    {\\mu}{$\mu$}1
    {\\nu}{$\nu$}1
    {\\xi}{$\xi$}1
    {\\pi}{$\pi$}1
    {\\rho}{$\rho$}1
    {\\sigma}{$\sigma$}1
    {\\tau}{$\tau$}1
    {\\upsilon}{$\upsilon$}1
    {\\phi}{$\phi$}1
    {\\chi}{$\chi$}1
    {\\psi}{$\psi$}1
    {\\omega}{$\omega$}1
    {\\Gamma}{$\Gamma$}1
    {\\Delta}{$\Delta$}1
    {\\Theta}{$\Theta$}1
    {\\Lambda}{$\Lambda$}1
    {\\Xi}{$\Xi$}1
    {\\Pi}{$\Pi$}1
    {\\Sigma}{$\Sigma$}1
    {\\Upsilon}{$\Upsilon$}1
    {\\Phi}{$\Phi$}1
    {\\Psi}{$\Psi$}1
    {\\Omega}{$\Omega$}1
    {\\EOF}{\;}1
    ,
  sensitive=true,  % case sensitive
  morecomment=[l]//,%
  morecomment=[l]{--},%
  morecomment=[s]{/*}{*/},%
  morestring=[b]",
  numbers=none,
  firstnumber=1,
  numberstyle=\tiny,
  stepnumber=2,
  basicstyle=\scriptsize\ttfamily,
  commentstyle=\itshape,
  keywordstyle=\bfseries,
  ndkeywordstyle=\bfseries,
}

\lstset{ % General setup for the package
	language=alloy,
	basicstyle=\small\sffamily,
	numbers=left,
    backgroundcolor=\color{lightgray},
 	numberstyle=\tiny,
	frame=tb,
	tabsize=4,
	columns=fixed,
	showstringspaces=false,
	showtabs=false,
	keepspaces,
	commentstyle=\color{red},
	keywordstyle=\color{blue}
}
% End of Alloy highlights

%% Sets page size and margins
\usepackage[a4paper,top=3cm,bottom=2cm,left=3cm,right=3cm,marginparwidth=1.75cm]{geometry}

%% Useful packages
\usepackage{amsmath}
\usepackage{graphicx}
\usepackage{verbatim}
\usepackage[colorinlistoftodos]{todonotes}
\usepackage[colorlinks=true, allcolors=black]{hyperref}

\title{Requirements Analysis and Specifications Document}
\author{Software Engineering 2}
\date{November 13,2016}

\begin{document}
\maketitle

\begin{figure}[h]
  \centering
  \includegraphics[width=300 pt]{resources/polimi.png}
  \label{fig:polimi}
\end{figure}

\begin{minipage}{0.4\textwidth}
\begin{flushleft} \large
\emph{Authors:}\\
B \textsc{A}\\
B \textsc{A}\\
B \textsc{A}
\end{flushleft}
\end{minipage}

\pagenumbering{gobble}
\newpage
\pagenumbering{arabic}

\tableofcontents

\newpage

\section{Introduction}
bla bla
\subsection{Goals}

%\textbf{Clients :}
\begin{enumerate}
\item A client can register to the system by providing an e-mail, any valid payment method and a photo of his driving license.

\item A client can log in in to system.

%\item 2 The system ,after verifies that the information of a new user are valid, send them a password and allow the user for rent a car.
%\item 2 bis : The system complete the registration and send back a password only if the user can be able to drive the proposed cars.

\item Clients can display available cars with the remaining charge near their current location (or a chosen address) with a maximum specified distance.Clients can display safe areas on a map and charging stations.

\item If Clients don’t have another active reservation they can choose a car that fit most their needs and reserve it for up one hour.

\item A client can start the renting opening a car that has reserved previously when he/she is in the near by.

\item During the rent a client can display the amount of money charged.%we must define "rent" in the glossary  

\item Guarantee as many available cars as possible encouraging clients to have a virtuous and eco-friendly behavior applying fees and discounts. %TODO : glossary good behavior

\item A client can leave the car in any safe area.Allows the client to end the rental.

\item Clients can report eventual damages made by themselves or by users that have driven the car before.

\item Clients can set the money saving option that suggest a charging station based on the car distribution and the destination selected by the client.

%\item Clients can communicate eventual car accidents or malfunctions.

\end{enumerate}

\begin{comment}

\textbf{Operator: }
\begin{enumerate}

\item Operators can see the status, charge level and location about all the cars.

\item Operators can open cars and drive them without paying.

\item Operators can set a car not available for maintenance proposes.   

\item Operators can display a list of cars that have a percentage of charge lower than 20\%.

\end{enumerate}
\end{comment}

\newpage

\begin{comment}
Which user groups are supported by the system to perform their work?
Which user groups execute the system’s main functions?
Which user groups perform secondary functions such as maintenance and administration?
With what external hardware or software system will the system-to-be interact?
What are the primary tasks that the system needs to perform?
What data will the actor create, store, change, remove or add in the system?
What external changes does the system need to know about?
What changes or events will the actor of the system need to be informed about?
\end{comment}

\section{Scenarios}
%G1-registration
\subsection {Registration Scenario}
Alice heard from his friend Charles that there is a new eco-friendly car sharing service in her town and she wants to use it. She installs the application PowerEnJoy on her device, opens the registration section and fills in the form with her personal data, e-mail and a credit card number. Then, she makes a photo of her driving license and uploads it. The system receives this information and verifies if it is valid by making a request to the national transport authority. So, the system sends back to Alice her password and pin, and allows her to rent a car.

%G2-G3 - Log in and display car information
\subsection{Log in and display car information Scenario}
John has to make a long travel to a venue called 'The wall' and he finds out that the only reasonable way to get there is by using a car sharing service. The only one for which he has registered is PowerEnJoy, but he has never used it before and John does know nothing about that place. In order to plan his route he logs in the web application and sees the available cars, the charging levels, the charging stations near his position and he discovers that 'The wall' is a safe area. 

%G4-reservation
\subsection{Reservation Scenario}
Bob needs to reach the airport at 3 am but there are no public transportation at that hour and the taxis are too expensive for him. So he decides to use the PowerEnJoy application from his smartphone. Once found the closest available car on the map, he reserves it and starts moving to the car location.

%G5-opening \rental G6 - display money charged 
\subsection{Opening / Rental scenario}
After 10 minutes of walk, Bob is in front of the car, he opens it using the app and he puts the luggage in the trunk. After inserting the pin, he starts the engine and selects the destination. At this point the rent starts and Bob makes his way to the airport. While Bob is driving to the airport, he can check how much he is spending in the car display.

%G7 ecology G8 
\subsection{Ecology Scenario}
Charles loves the PowerEnJoy philosophy of respecting nature and he uses the service every day to go work, there is a power station near his office so each time he parks the car there and plugs it into the power grid to get the discount. The office, is located in a safe zone so whenever the power station is full he leaves the car in the nearby and he always checks that the system closes the car before leaving.

%G9 - reporting abuse 
\subsection{Reporting dirty car Scenario}
Hahn is a solo singer and he wants to make a walk in the park with his dog Chew and his child Loren. At the end of the day they are very tired and they don't want to come back home walking so they decide to use PowerEnJoy. The system finds that there is a car nearby, in front of an ice-cream shop. When they arrive to the car Loren cannot resist to the temptation of the ice-cream so he starts crying and Hahn to makes him stops decides to buy him an ice-cream, but in the meanwhile Chew is playing in the mud because he is feeling alone. Hahn does not want to make the car dirty but he sees that Jabba, a friend of him, is coming. Han has to pay Jabba a conspicuous amount of money and since Hahn doesn't have that money yet he decides to jump in the car. Hahn is used to driving very fast and near a traffic light he slams on the brakes and Loren drops the ice-cream, Chew jumps to the front seat and eats it. After arriving at the destination they leave the car all dirty. After a while Anakin reserves the same car but he finds it dirty so he uses the function provided by the application and signals the abuse. PowerEnJoy applies a fee to Hahn and blocks him for two months.

%G10 - Saving Mode
\subsection{Saving Mode Scenario}
Donnie, Ester, Frank and Gabriela are students and they want to travel in the cheapest possible way. Today is Gabriela's birthday and they want to go to the mall to celebrate and since Gabriela is eco-friendly they decide to go with an electric car. Frank has a PowerEnJoy account so they decide to use the service and since it's cheaper they use the money saving option. The application suggests them the nearest car with enough charge to reach the mall, then it gives information to get to the car and once opened it the display inside the car suggests the charging stations tha are nearest to the mall. When they arrive they plug the car in the charging grid. The system recognizes that they plugged the car, they left the car with more than 50\% of residual charge, and that there were 4 passengers, so it applies 10\% + 30\% + 20\% of discount.

\begin{abstract}
Your abstract.
\end{abstract}

\section{Some examples to get started}

\subsection{Alloy}
\lstinputlisting{resources/alloy.txt}

\begin{comment}
\end{}
\begin{lstlisting}
// A file system object in the file system
abstract sig FSObject { }
sig File, Dir extends FSObject { }

// A File System
sig FileSystem {
  root: Dir,
  live: set FSObject,
  contents: Dir lone-> FSObject,
  parent: FSObject ->lone Dir
}{
  // root has no parent
  no root.parent
  // live objects are reachable from the root
  live in root.*contents
  // parent is the inverse of contents
  parent = ~contents
}

pred example { }

run example for exactly 1 FileSystem, 4 FSObject
\end{lstlisting}
\end{comment}

\subsection{How to include Figures}

\begin{figure}[h]
\centering
\includegraphics[width=0.3\textwidth]{resources/polimi.png}
\caption{\label{fig:frog}This was uploaded via the project menu.}
\end{figure}

\subsection{How to add Tables}

Use the table and tabular commands for basic tables --- see Table~\ref{tab:widgets}, for example. 

\begin{table}[h]
\centering
\begin{tabular}{l|r}
Item & Quantity \\\hline
Widgets & 42 \\
Gadgets & 13
\end{tabular}
\caption{\label{tab:widgets}An example table.}
\end{table}

\subsection{How to write Mathematics}

\LaTeX{} is great at typesetting mathematics. Let $X_1, X_2, \ldots, X_n$ be a sequence of independent and identically distributed random variables with $\text{E}[X_i] = \mu$ and $\text{Var}[X_i] = \sigma^2 < \infty$, and let
\[S_n = \frac{X_1 + X_2 + \cdots + X_n}{n}
      = \frac{1}{n}\sum_{i}^{n} X_i\]
denote their mean. Then as $n$ approaches infinity, the random variables $\sqrt{n}(S_n - \mu)$ converge in distribution to a normal $\mathcal{N}(0, \sigma^2)$.

\subsection{How to add Citations and a References List}

You can upload a \verb|.bib| file containing your BibTeX entries, created with JabRef; or import your \href{https://www.overleaf.com/blog/184}{Mendeley}, CiteULike or Zotero library as a \verb|.bib| file. You can then cite entries from it, like this: \cite{greenwade93}. Just remember to specify a bibliography style, as well as the filename of the \verb|.bib|.

You can find a \href{https://www.overleaf.com/help/97-how-to-include-a-bibliography-using-bibtex}{video tutorial here} to learn more about BibTeX.

We hope you find Overleaf useful, and please let us know if you have any feedback using the help menu above --- or use the contact form at \url{https://www.overleaf.com/contact}!

\bibliographystyle{alpha}
\bibliography{sample}

\end{document}