\documentclass[a4paper]{article}

%% Language and font encodings
\usepackage[english]{babel}
\usepackage[utf8x]{inputenc}
\usepackage[T1]{fontenc}


% From https://gist.github.com/timvdalen/3796300
% From https://hackage.haskell.org/package/alloy-1.2.0/src/tutorial/tutorial.tex
\usepackage{color}
\usepackage{listings} % For Alloy code highlights
% alloy.sty
% Alloy mode for the LaTeX listings package.
% This is public domain

\lstdefinelanguage{alloy}{
  keywords={%
      assert, pred, all, no, lone, one, some, check, run,
      but, let, implies, not, iff, in, and, or, set, sig, Int, int,
      if, then, else, exactly, disj, fact, fun, module, abstract,
      extends, open, none, univ, iden, seq,
  },
  literate=%
    {:}{$\colon$}1
    {|}{$\bullet$}1
    {==}{$=$}1
    {=}{$=$}1
    {!=}{$\neq$}1
    {&&}{$\land$}1
    {||}{$\lor$}1
    {<=}{$\le$}1
    {>=}{$\ge$}1
    {all}{$\forall$}1
    {exists}{$\exists$}1
    {!in}{$\not\in$}1
    {\\in}{$\in$}1
    {=>}{$\implies$}2
    % the following isn't actually Alloy, but it gives the option to produce nicer latex
    {|=>}{$\Rightarrow$}2
    {<=set}{$\subseteq$}1
    {+set}{$\cup$}1
    {*set}{$\cap$}1
    {==>}{$\Longrightarrow$}3
    {<==>}{$\Longleftrightarrow$}4
    {...}{$\ldots$}1
    {\\hl}{$\hline$}1
    {\\alpha}{$\alpha$}1
    {\\beta}{$\beta$}1
    {\\gamma}{$\gamma$}1
    {\\delta}{$\delta$}1
    {\\epsilon}{$\epsilon$}1
    {\\zeta}{$\zeta$}1
    {\\eta}{$\eta$}1
    {\\theta}{$\theta$}1
    {\\iota}{$\iota$}1
    {\\kappa}{$\kappa$}1
    {\\lambda}{$\lambda$}1
    {\\mu}{$\mu$}1
    {\\nu}{$\nu$}1
    {\\xi}{$\xi$}1
    {\\pi}{$\pi$}1
    {\\rho}{$\rho$}1
    {\\sigma}{$\sigma$}1
    {\\tau}{$\tau$}1
    {\\upsilon}{$\upsilon$}1
    {\\phi}{$\phi$}1
    {\\chi}{$\chi$}1
    {\\psi}{$\psi$}1
    {\\omega}{$\omega$}1
    {\\Gamma}{$\Gamma$}1
    {\\Delta}{$\Delta$}1
    {\\Theta}{$\Theta$}1
    {\\Lambda}{$\Lambda$}1
    {\\Xi}{$\Xi$}1
    {\\Pi}{$\Pi$}1
    {\\Sigma}{$\Sigma$}1
    {\\Upsilon}{$\Upsilon$}1
    {\\Phi}{$\Phi$}1
    {\\Psi}{$\Psi$}1
    {\\Omega}{$\Omega$}1
    {\\EOF}{\;}1
    ,
  sensitive=true,  % case sensitive
  morecomment=[l]//,%
  morecomment=[l]{--},%
  morecomment=[s]{/*}{*/},%
  morestring=[b]",
  numbers=none,
  firstnumber=1,
  numberstyle=\tiny,
  stepnumber=2,
  basicstyle=\scriptsize\ttfamily,
  commentstyle=\itshape,
  keywordstyle=\bfseries,
  ndkeywordstyle=\bfseries,
}

\lstset{ % General setup for the package
	language=alloy,
	basicstyle=\small\sffamily,
	numbers=left,
    backgroundcolor=\color{lightgray},
 	numberstyle=\tiny,
	frame=tb,
	tabsize=4,
	columns=fixed,
	showstringspaces=false,
	showtabs=false,
	keepspaces,
	commentstyle=\color{red},
	keywordstyle=\color{blue}
}
% End of Alloy highlights

%% Sets page size and margins
\usepackage[a4paper,top=3cm,bottom=2cm,left=3cm,right=3cm,marginparwidth=1.75cm]{geometry}

%% Useful packages
\usepackage{amsmath}
\usepackage{graphicx}
\usepackage{verbatim}
\usepackage[colorinlistoftodos]{todonotes}
\usepackage[colorlinks=true, allcolors=black]{hyperref}

\title{Requirements Analysis and Specifications Document}
\author{Software Engineering 2}
\date{November 13,2016}

\begin{document}
\maketitle

\begin{figure}[h]
  \centering
  \includegraphics[width=300 pt]{resources/polimi.png}
  \label{fig:polimi}
\end{figure}

\emph{\\}
\emph{\\}
\emph{\\}
\emph{\\}
\emph{\\}
\emph{\\}
\emph{\\}
\emph{\\}
\emph{\\}
\emph{\\}

\begin{minipage}{0.4\textwidth}
\begin{flushleft} \large
\emph{Authors:}\\
Claudio Salvatore \textsc{Arcidiacono}\\
Antonio \textsc{Di Bello}\\
Denis \textsc{Dushi}
\end{flushleft}
\end{minipage}

\begin{minipage}{0.4\textwidth}

\end{minipage}

\pagenumbering{gobble}
\newpage
\pagenumbering{arabic}

\tableofcontents

\newpage

\section{Introduction}
bla bla
\subsection{Goals}

\begin{enumerate}
\item A client can register to the system by providing an e-mail, any valid payment method and a photo of his driving license.

\item A client can access to his profile.

\item Allows clients to obtain information about
the position of available cars, safe areas and charging stations; about the remaining charge of cars and the available plugs of the charging stations.

\item If Clients don’t have another active reservation they can choose a car that fit most their needs and reserve it for up one hour.

\item A client can start the renting opening a car that has reserved previously when he/she is in the near by.

\item During the rent a client can display the amount of money charged.

\item Guarantee as many available cars as possible encouraging clients to have a virtuous and eco-friendly behavior applying fees and discounts.

\item Allows the client to end the rental and leave the car in any safe area.

\item Clients can report eventual damages made by themselves or by users that have driven the car before.

\item Clients can set the money saving option that suggest a charging station based on the car distribution and the destination selected by the client.

%\item Clients can communicate eventual car accidents or malfunctions.

\end{enumerate}


\subsection{Domain assumption}
We suppose that these properties hold in the analyzed world :
\begin{itemize}
\item All cars have a stable GPS signal.
\item Clients shares their position when they use the application.
\item PowerEnJoy always has updated data about all the cars and the power stations.
\item The power grid always guarantees electric power.
\item The only method to enter in a car is by the doors.
\item Client knows how to drive a car.
\item The company cars have at least 4 seats.
\item The streets are correctly mapped in the system.
\item All cars have stable internet connection.
\item The charging sensor reflects the real charge of the car in real-time.
\item The real charge of parked cars stays the same over time. 
\end{itemize}


\newpage
\section{Requirements}
\begin{enumerate}
\item A client can register to the system by providing an e-mail, any valid payment method and a photo of his driving license.
\begin{enumerate}
\item The system must be able to check the validity of the payment information.
\item The system must be able to check if the uploaded documents are in the right format.
\item The system must be able to comunicate with (ente trasporti e boh) in order to obtain user profile informations from id number and licence number. %(boh??Da riscrivere)
\item The system must allow the registration only if ID ,licence and personal informations are related to the same person.
\item The system must  implement a retrieve password mechanism.
\item The system should not allow to complete the registration process to an already registered user.
\item The system should not allow different users to register with the same email.
\end{enumerate}

\item A client can log in in to system.
\begin{enumerate}
\item The application an the web application should provide the log in function.
\item The system should allow the client to rent a car using his profile after effectuated the log in.
\item The system must guarantee that none can access to a personal area of an user without the right credentials.
\item The system must be able to check if the credentials inserted by the user are correct.
\end{enumerate}

\item Allows clients to display a map in which are indicated the available cars and their charge percentage, the safe areas and the charging stations.
\begin{enumerate}
\item The system should knows the correct location of available cars.
\item The system should updates in real time the information about the charging stations.
\item The system must offers a view of a map in which are represented the available cars, the charging station and the boundary of the safe area.
\item The system must know the position of clients that have opened the application.
\item The system should be able to send information about available cars and charging stations in a specified area.
\end{enumerate}


\item If clients don’t have another active reservation they can choose a car that fit most their needs and reserve it for up one hour.
\begin{enumerate}
\item  
\end{enumerate}


\item A client can start the renting opening a car that has reserved previously when he/she is in the near by.
\begin{enumerate}
\item  
\end{enumerate}

\item During the rent a client can display the amount of money charged.
\begin{enumerate}
\item  
\end{enumerate}

\item Guarantee as many available cars as possible encouraging clients to have a virtuous and eco-friendly behavior applying fees and discounts.
\begin{enumerate}
\item  
\end{enumerate}

\item Allows the client to end the rental and leave the car in any safe area.
\begin{enumerate}
\item  
\end{enumerate}

\item Clients can report eventual damages made by themselves or by users that have driven the car before.
\begin{enumerate}
\item  
\end{enumerate}

\item Clients can set the money saving option that suggest a charging station based on the car distribution and the destination selected by the client.
\begin{enumerate}
\item  
\end{enumerate}

\end{enumerate}

\newpage

\section{Glossary}
\textbf{Guest}
A Guest is a user that access for the first time to the platform.\\
\textbf{Client}
A registered User.\\
\textbf{User:}
a client or a guest.\\
\textbf{Virtuous  behavior}
asdasdasdad.\\
\textbf{Rent:}
\\
\textbf{Map Information:}
information displayed on the map...\\
\textbf{Not valid data:}
during the registration ( ex: email yet used by an other user )\\
\textbf{Credentials:}
email and password in the registration area.\\
\textbf{Valid Credentials:}
an email registered and its password.\\
\textbf{Car information Menu:}
The menu that is displayed when a client select a car on the map. In this menu are displayed the information about the car and a button that can be used to reserve the car.\\
\textbf{Safe Areas:} 
\\
\textbf{Valid Payment methods:} 
\\
\textbf{User authentication:}
\\

\section{Actor}

\newpage

\section{Scenarios}
%G1-registration
\subsection {Registration Scenario}
Alice heard from his friend Charles that there is a new eco-friendly car sharing service in her town and she wants to use it. She installs the application PowerEnJoy on her device, opens the registration section and fills in the form with her personal data, e-mail and a credit card number. Then, she makes a photo of her driving license and uploads it. The system receives this information and verifies if it is valid by making a request to the national transport authority. So, the system sends back to Alice her password and pin, and allows her to rent a car.

%G2-G3 - Log in and display car information
\subsection{Log in and display car information Scenario}
John has to make a long travel to a venue called 'The wall' and he finds out that the only reasonable way to get there is by using a car sharing service. The only one for which he has registered is PowerEnJoy, but he has never used it before and John does know nothing about that place. In order to plan his route he logs in the web application and sees the available cars, the charging levels, the charging stations near his position and he discovers that 'The wall' is a safe area. 

%G4-reservation
\subsection{Reservation Scenario}
Bob needs to reach the airport at 3 am but there are no public transportation at that hour and the taxis are too expensive for him. So he decides to use the PowerEnJoy application from his smartphone. Once found the closest available car on the map, he reserves it and starts moving to the car location.

%G5-opening \rental G6 - display money charged 
\subsection{Opening / Rental scenario}
After 10 minutes of walk, Bob is in front of the car, he opens it using the app and he puts the luggage in the trunk. After inserting the pin, he starts the engine and selects the destination. At this point the rent starts and Bob makes his way to the airport. While Bob is driving to the airport, he can check how much he is spending in the car display.

%G7 ecology G8 
\subsection{Ecology Scenario}
Charles loves the PowerEnJoy philosophy of respecting nature and he uses the service every day to go work, there is a power station near his office so each time he parks the car there and plugs it into the power grid to get the discount. The office, is located in a safe zone so whenever the power station is full he leaves the car in the nearby and he always checks that the system closes the car before leaving.

%G9 - reporting abuse 
\subsection{Reporting dirty car Scenario}
Hahn is a solo singer and he wants to make a walk in the park with his dog Chew and his child Loren. At the end of the day they are very tired and they don't want to come back home walking so they decide to use PowerEnJoy. The system finds that there is a car nearby, in front of an ice-cream shop. When they arrive to the car Loren cannot resist to the temptation of the ice-cream so he starts crying and Hahn to makes him stops decides to buy him an ice-cream, but in the meanwhile Chew is playing in the mud because he is feeling alone. Hahn does not want to make the car dirty but he sees that Jabba, a friend of him, is coming. Han has to pay Jabba a conspicuous amount of money and since Hahn doesn't have that money yet he decides to jump in the car. Hahn is used to driving very fast and near a traffic light he slams on the brakes and Loren drops the ice-cream, Chew jumps to the front seat and eats it. After arriving at the destination they leave the car all dirty. After a while Anakin reserves the same car but he finds it dirty so he uses the function provided by the application and signals the abuse. PowerEnJoy applies a fee to Hahn and blocks him for two months.

%G10 - Saving Mode
\subsection{Saving Mode Scenario}
Donnie, Ester, Frank and Gabriela are students and they want to travel in the cheapest possible way. Today is Gabriela's birthday and they want to go to the mall to celebrate and since Gabriela is eco-friendly they decide to go with an electric car. Frank has a PowerEnJoy account so they decide to use the service and since it's cheaper they use the money saving option. The application suggests them the nearest car with enough charge to reach the mall, then it gives information to get to the car and once opened it the display inside the car suggests the charging stations tha are nearest to the mall. When they arrive they plug the car in the charging grid. The system recognizes that they plugged the car, they left the car with more than 50\% of residual charge, and that there were 4 passengers, so it applies 10\% + 30\% + 20\% of discount.

\newpage
\section{Use Cases Description}
\subsubsection{Register to PowerEnJoy}
%
\textbf{Actors:}
Guest\\
%
\textbf{Entry Conditions:}
A guest enters in the registration section.\\ 
%
\textbf{Flow of events:}
\begin{enumerate}
\item The guest fills the form with his personal data, the number of his ID card and the number of the driving license. 
\item The guest uploads driver license picture.
\item  The guest fills the payment information form with his credit or prepaid card's information. 
\item PowerEnJoy verifies if the data are valid and sends an email to the guest with his password and a pin necessary to start the rental. 
\item PowerEnJoy registers the guest as a new client. 
\end{enumerate}
%
\textbf{Exit Conditions:}
Guest successfully ends registration process and becomes a Client. From now on he can log in to the application using his credential and use the PowerEnJoy service. \\ 
%
\textbf{Exceptions:}
\begin{itemize}
\item If the guest inserts not valid data the system highlights the not valid fields and doesn't allow him to complete the registration.
\item If the guest inserts an already used Email PowerEnJoy doesn't allow him to complete the registration.
\item The system does not allow an existing client to register a second time.
\end{itemize}
%
\textbf{Special Requirements:}
\begin{itemize}
\item After the registration the client must be able to use the service after no more than 5 minutes.
\item After that the guest has filled the form, he must receive an email feedback with the password and pin within two hours.
\item The communication of sensible data is made throw a protected channel.
\end{itemize}


\subsubsection{Log-in}
%
\textbf{Actors:}
Client\\
%
\textbf{Entry Conditions:}
A client enters in the log-in section.\\
%
\textbf{Flow of events:}
\begin{enumerate}
\item The client enters his credentials.
\item PowerEnJoy verifies if credentials are valid.
\item PowerEnJoy redirects the client to the map section.
\end{enumerate}
%
\textbf{Exit Conditions:}
Client is logged in to the system.\\
%
\textbf{Exceptions:}
\begin{itemize}
\item The client inserts not valid credentials. PowerEnJoy notifies him an error and allows him to enter his email and password again.
\item If the client is a blocked client the system notifies him an error and doesn't allow him to log-in.
\end{itemize}
%
\textbf{Special Requirements:}
\begin{itemize}
\item The system should provide a feedback in 10 seconds.
\end{itemize}


\subsubsection{Display a map information}
%
\textbf{Actors:}
User\\
%
\textbf{Entry Conditions:}
Entering in the map section. \\
%
\textbf{Flow of events:}
\begin{enumerate}
\item User sends his GPS position or selects an address on the map, and specifies a maximum distance from this point.
\item PowerEnJoy displays the data about the part of map requested: 
\begin{itemize}
\item location of available cars
\item location of charging areas and number of available charging spots
\item boundaries of safe areas 
\end{itemize}
\item User requests information about one selected car.
\item PowerEnJoy sends back the remaining charge percentage and the license plate number of the selected car .
\item User can see the requested information in a car information menu.
\end{enumerate}
%
\textbf{Exit Conditions:}
Exit from the map section, or a start of a reservation. \\
%
\textbf{Exceptions:}
\begin{itemize}
\item If User doesn't obtain a valid GPS signal, PowerEnJoy uses a default position like the center of his preferred city.
\item If User requests information about a car that is not available anymore, PowerEnJoy doesn't send any information about this car, and all the positions of displayed cars are refreshed.
\item If User selects an address out of city, PowerEnJoy uses a default position.
\end{itemize}
%
\textbf{Special Requirements:}
\begin{itemize}
\item The map will refresh every 10 seconds.
\end{itemize}


\subsubsection{Reserve a car}
%
\textbf{Actors:}
Logged Client\\
%
\textbf{Entry Conditions:}
A client clicks the Reserve button in the car information menu.\\
%
\textbf{Flow of events:}
\begin{enumerate}
\item PowerEnJoy sets the car as reserved.
\item The client receives the confirmation of the reservation.
\item The client checks the remaining time of the reservation.  
\end{enumerate}
%
\textbf{Exit Conditions:}
The car is no longer available, the client has now a rent in progress.\\
%
\textbf{Exceptions:}
\begin{itemize}
\item The client has already canceled a reservation on the same car in the last 15 minutes. 
\item The car is no longer available.
\end{itemize}
In all of these cases the reservation process is not performed and the user is redirected to the map section. \\
%
\textbf{Special Requirements:}
\begin{itemize}
\item The system should provide a feedback in 10 seconds.
\end{itemize}


\subsubsection{Open a car}
%
\textbf{Actors:}
Logged Client \\
%
\textbf{Entry Conditions:}
Client has done a reservation. \\
%
\textbf{Flow of events:}
\begin{enumerate}
\item Client moves to a car location, within an hour from the reservation.
\item Once he has arrived, he can push the open button from the application.
\item PowerEnJoy verifies the client position and releases the car doors lock.
\item Client opens the car within a minute.
\end{enumerate}
%
\textbf{Exit Conditions:}
Client starts the rent. \\
%
\textbf{Exceptions:}
\begin{itemize}
\item If the client doesn't reach the car within one hour, PowerEnJoy deletes the reservation and applies a fee of 1 Euro.
\item If the client tries to open the car while he is farther more than 15 meters from the car or he has not GPS signal, PowerEnJoy doesn't unlock the car door.
\end{itemize}
%
\textbf{Special Requirements:}
\begin{itemize}
\item PowerEnJoy unlocks the car within a minute from the request.
\end{itemize}



\subsubsection{Start the rent}
%
\textbf{Actors:}
Logged Client \\
%
\textbf{Entry Conditions:}
The system has opened the car.\\
%
\textbf{Flow of events:}
\begin{enumerate}
\item The Client inserts the pin.
\item The system after checking the correctness of the pin enables the possibility to turn on the engine.
\item The Client turns on the engine.
\item The Client selects his destination.
\item The car screen displays the current cost and route information.
\item The Client starts moving to his destination.
\item The Client sees in real time how much he is spending.
\end{enumerate}
\textbf{Exit Conditions:}
The Client is driving the rented car.\\
%
\textbf{Exceptions:}
\begin{itemize}
\item If the charge of the car is lower than 15\% the car system signals to park the car immediately.
\item If the client continues driving until the car battery is empty the system applies a fee of 100 euros to the client.
\item If the rent takes more than 1 day the system signals the case to the operators system.
\end{itemize}
%
\textbf{Special Requirements:}
There are no special Requirements.


\subsubsection{End the rent}
%
\textbf{Actors:}
Logged Client \\
%
\textbf{Entry Conditions:}
Client reaches the destination. \\
%
\textbf{Flow of events:}
\begin{enumerate}
\item Client parks the car in a safe area and turns off the engine.
\item Client pushes the button to end the rent and exits the car.
\item PowerEnJoy receives the request, checks if the car is parked in a safe area and, after one minute, locks the doors of the car and terminates the rent.
\item PowerEnJoy charges the client after 5 minutes to give him the possibility to plug the car.
\end{enumerate}
%
\textbf{Exit Conditions:}
The car is set as available. \\
%
\textbf{Exceptions:}
\begin{itemize}
\item If the client tries to leave the car out of a safe area, the system doesn't allow him to end the rent.
\item If the client doesn't have enough money to pay the rent, PowerEnJoy disables the client account until the debt is absolved.
\item If the client doesn't exit from the car after one minute, the system doesn't lock the car doors and doesn't terminate the rent.
\end{itemize}
%
\textbf{Special Requirements:}
There are no special Requirements.



\subsubsection{Plug the car}
%
\textbf{Actors:}
Logged Client \\
%
\textbf{Entry Conditions:}
The Client parks the car in a charging station with free plugs and ends the rent. \\
%
\textbf{Flow of events:}
\begin{enumerate}
\item The Client opens the charging slot of the car.
\item The Client plugs the plug to the car.
\item The car system applies a discount of 30\% to the last rent.
\end{enumerate}
%
\textbf{Exit Conditions:}
The car is correctly plugged-in and is recharging.\\
%
\textbf{Exceptions:}
If the client plugs the car after 5 minutes, the system doesn't apply the discount.\\
%
\textbf{Special Requirements:}
There are no Special Requirements.



\subsubsection{Report damaged/dirty car}
%
\textbf{Actors:}
Logged Client \\
%
\textbf{Entry Conditions:}
Client is in front of a damaged or dirty car he has reserved.\\
%
\textbf{Flow of events:}
\begin{enumerate}
\item A Client decides to report the conditions of the car by clicking the "Report damaged/dirty car" button on the app.
\item PowerEnJoy redirects the client to a form where he has to indicate:
\begin{itemize}
\item Type of Abuse : "Damaged" or "Dirty" 
\item Gravity of the problem : from 0 to 5
\end{itemize}
\item PowerEnJoy signals the case to the operators system.
\item PowerEnJoy proposes to the client two alternatives :
\begin{itemize}
\item "Continue rent"
\item "End rent"
\end{itemize}
\item The client selects the desired choice.
\end{enumerate}
%
\textbf{Exit Conditions:}
The car is set unavailable.  \\
\textbf{Exceptions:}
 There are not exceptions. \\
\textbf{Special Requirements:}
 There are not Special Requirements\\

\subsubsection{Select money saving option}
%
\textbf{Actors:}
Logged Client \\
%
\textbf{Entry Conditions:}
The Client selects the money saving option on the car display.\\
%
\textbf{Flow of events:}
\begin{enumerate}
\item The system estimates which is the most suitable charging station based on the city car distribution and the location chosen by the client. 
\item The car system suggests a possible charging station.
\item The client selects the suggested charging station.
\end{enumerate}
%
\textbf{Exit Conditions:}
The car system starts the navigation to the chosen charging station. \\
%
\textbf{Exceptions:}
\begin{itemize}
\item If the client doesn't like the suggested charging station he can undo the action and proceed with the normal mode.
\end{itemize}
%
\textbf{Special Requirements:}
\begin{itemize}
\item The system should provide feedback within 10 seconds.
\end{itemize}


\newpage
\section{Alloy}
\lstinputlisting{resources/alloy.txt}



\section{Some examples to get started}

\subsection{How to include Figures}

\begin{figure}[h]
\centering
\includegraphics[width=0.3\textwidth]{resources/polimi.png}
\caption{\label{fig:frog}This was uploaded via the project menu.}
\end{figure}

\subsection{How to add Tables}

Use the table and tabular commands for basic tables --- see Table~\ref{tab:widgets}, for example. 

\begin{table}[h]
\centering
\begin{tabular}{l|r}
Item & Quantity \\\hline
Widgets & 42 \\
Gadgets & 13
\end{tabular}
\caption{\label{tab:widgets}An example table.}
\end{table}

\subsection{How to write Mathematics}

\LaTeX{} is great at typesetting mathematics. Let $X_1, X_2, \ldots, X_n$ be a sequence of independent and identically distributed random variables with $\text{E}[X_i] = \mu$ and $\text{Var}[X_i] = \sigma^2 < \infty$, and let
\[S_n = \frac{X_1 + X_2 + \cdots + X_n}{n}
      = \frac{1}{n}\sum_{i}^{n} X_i\]
denote their mean. Then as $n$ approaches infinity, the random variables $\sqrt{n}(S_n - \mu)$ converge in distribution to a normal $\mathcal{N}(0, \sigma^2)$.

\subsection{How to add Citations and a References List}

You can upload a \verb|.bib| file containing your BibTeX entries, created with JabRef; or import your \href{https://www.overleaf.com/blog/184}{Mendeley}, CiteULike or Zotero library as a \verb|.bib| file. You can then cite entries from it, like this: \cite{greenwade93}. Just remember to specify a bibliography style, as well as the filename of the \verb|.bib|.

You can find a \href{https://www.overleaf.com/help/97-how-to-include-a-bibliography-using-bibtex}{video tutorial here} to learn more about BibTeX.

We hope you find Overleaf useful, and please let us know if you have any feedback using the help menu above --- or use the contact form at \url{https://www.overleaf.com/contact}!

\bibliographystyle{alpha}
\bibliography{sample}

\end{document}