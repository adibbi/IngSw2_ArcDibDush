In the section below we will use function points and COCOMO II approaches in order to provide an estimation of the expected size, cost and required effort for \textbf{PoweEnJoy}.



\subsection{Size estimation: function points}
The size estimation will be estimated using the \textbf{Function Points} approach, spotting at first what are the main functionalities of our system and how many lines of code we aspect to be written in JEE. According to the function points approach, the LOC can be estimated using the following formula:

\begin{center}
$LOC =  FP \cdot AVC $
\end{center}

where:
\begin{itemize}
\item LOC = Lines of code
\item FP = Function Points
\item AVC = language specific constant, for JEE is between 46 for an lower bound and 67 for an upper bound
\end{itemize}
In order to estimate the overall function points, this approach needs a first classification of our main files into 2 categories: 
\begin{itemize}
\item Internal Logic files : homogeneous set of data used and managed by the application .
\item External Interface files:homogeneous set of data used by the application but generated and maintained by other applications.
\end{itemize}
and a classification of the main functionalities into 3 main categories:
\begin{itemize}
\item External Input: Elementary operation to elaborate data coming from the external environment.
\item External Output:Elementary operation that generates data for the external environment.
\item External Inquiry:Elementary operation that involves input and output, Without significant elaboration of data from logic files.
\end{itemize}
Once we have identified  the most relevant functions of our system, we attribute a  complexity to each of them and based on the category to which it appertain we  assign some function points.
The function points have been obtained from statistical studies of past projects.
The  weights that we will use for each category will be:\\
\begin{center}
\begin{tabular}{ c c c c  }
\hline
 &\multicolumn{3}{ c }{Complexity Weight} \\\hline
 \textit{Function Type} & \textit{Low}&\textit{Average}&\textit{High}\\\hline
 Internal Logic Files & 7 & 10&15\\
External Interface Files & 5 & 7&10\\
External Input & 3& 4&6\\
External Output & 4& 5&7\\
External Inquiries & 3& 4&6\\

\hline

\end{tabular}
\end{center}
\newpage
\subsubsection{Internal Logic Files (ILF)}
PowerEnJoy need to stores a big amount of data about users, cars and his  assets, so it will use several ILFs. Here below we will see in detail the most relevant.
\begin{itemize}

\item First of all the system will store data about its users (name, surname, email, address, fiscalcode, driving license code and picture ,  ID number, cellphone number) and since this entity will have several attributes  and it will contain around 200.000 entries (considering that the system will run in a city like Milan, where other competitors have similar numbers) we can attribute to this ILF an \textbf{Average} complexity.\\
\item For each client  the system will store data about his payment information, but since this entity has few attributes and it is related 1 to 1  just with the client data, and since these data will be inserted and uploaded rarely we can consider its complexity as \textbf{Low}.\\
\item For each client we will need to store his login data, for the same reasoning as above we will consider this entity complexity as \textbf{Low}.\\
\item The main function of our system is to allow users to reserve and rent cars so the main data that we will store will concern reservations and rents. Assuming that an average client will do 4-5 rents we estimate that the system must be able to store 800.000 - 1M entries. During the day there are some hours were our users will use the system more than during others so we will have a huge traffic of insert and delete that our system must be able to handle in parallel. Furthermore this entities are linked with a lot of other structures in our system so, given the nature of them we can estimate  their complexity as \textbf{High}.\\
\item The system also need to store information about cars, an entity that will be updated pretty often and that has several relations with other structures; but we assume that the system will handle between few hundreds and a thousand cars (so entries); compared with other entities they aren't that much, so  we can estimate the complexity of the ILF as \textbf{Low}.\\
\item The system has to provide a map that will contain information about cars,  were to park them and where to charge them so it has to store information about safe areas and charging stations.  Since these two entities contain few entries (around one hundred charging stations and  tens of safe areas) and that these tables will not be updated very often we can consider their complexity as \textbf{Low}.\\
\end{itemize}
The overall estimation for ILF is summarized in the following table:
\begin{center}
\begin{tabular}{|l l l|}
\hline
\multicolumn{1}{|c}{ILF}&Complexity&FPs\\
\hline
Clients data & Average & 10\\
Payment Information data & Low & 7\\
Login data &Low&7\\
Reservations & High & 15\\
Rents & High&15\\
Cars & Low & 7\\
Safe Areas & Low& 7 \\
Charging Stations & Low & 7\\\hline
\multicolumn{2}{|c}{Total}&75\\
\hline
\end{tabular}
\end{center}
\newpage

\subsubsection{External Interface Files (EIF)}
The external interface files  that the system relies on are the one coming from the external interfaces: the  transport authority, the civil registry and the payment gateway. The system communicates also with an external interface to the operator system but since there are no input data from that interface we will not consider it. Here we will analyze the three EIFs:
\begin{itemize}
\item During the registration phase the user has to input his driving license number; this data is used to call an external service provided by the transport authority and to verify that the driving license is valid for our purpose. This is an operation that happens rarely (once per client) and that involves only one structure ( client data ) using only one request, so we can consider this EIF as a \textbf{Low} complexity one. \\
\item  During the registration phase the user has to input his ID number, this data is used to call an external service provided by the civil registry  to verify the validity of the data inserted. As the one above this operation happens rarely (once per client) and involves only one structure ( client data ) using only one request, so we can consider this EIF as a \textbf{Low} complexity one. \\ 
\item After that a client has ended the rent the system start the payment process throw the payment gateway. From this process we acquire the receipts for the payment that has to be stored. Since this operation will take place pretty often and since these are sensible data and for this reason have to be stored in an high security data structure we can consider this EIF as an \textbf{High} complexity one.
\end{itemize}
Summarizing this function we can derive the following table:
\begin{center}
\begin{tabular}{|l l l|}
\hline
\multicolumn{1}{|c}{EIF}&Complexity&FPs\\
\hline
Driving License & Low& 5 \\
Personal Data & Low & 5\\
Payments Data& High&10\\\hline
\multicolumn{2}{|c}{Total}&20\\
\hline
\end{tabular}
\end{center}
\subsubsection{External Inputs(EI)}
The PowerEnjoy system has to interact with our clients for different purposes (In order to understand the complexity of these functions we will refer often to the components described in the components diagram of the design document):
\begin{itemize}
\item \textbf{Login/Logout:} This is an operation that requires only a component (UserController) and just with one entity, so we can classify this function as a \textbf{Low} complexity one.
\item \textbf{Registration:} This operation requires one component and the interaction with two external interfaces so we can estimate its complexity as \textbf{Average}.
\item \textbf{Reservation Request:} This operation requires the cooperation of a few components; we will have to verify if the car is available (CarController), if so we have to insert the reservation in the database, modify the status of the car (CarRemoteController and CarController) and send the feedback to the user so we can estimate its complexity as \textbf{Average}.
\item \textbf{Cancel Reservation:} This operation requires the same components as the one before so reasoning in the same way we can estimate its complexity as \textbf{Average}.
\item \textbf{Report Damaged Car:} This operation requires a complex function in the CarRemoteController that has to interact with the server system (using CarController) and then communicate with the operator system using an external interface, so given the number of components and the interaction throw an external interface we can estimate its complexity as \textbf{High}.
\item \textbf{Start rent:} this operation requires the cooperation of quite all the components of our system and it has to query the database changing the status of the rent and the reservation, so we can estimate this operation as an \textbf{High} complexity function.
\item \textbf{End rent:} This operation requires the cooperation of a lot of components: CarController for the car position, CarController to change the status of the car and check if the car is plugged, RentController to check information about the rent and others. It also has to start the payment process throw an external interface and it has to query the database to update the rent data. We can consider it an \textbf{High} complexity function.
\item \textbf{Pin Insertion:} After that a client enters in a car and wants to start the engine he has to authenticate  himself inserting the pin in the car system. This operation is just a simple check that  requires the cooperation of few components(CarController, CarRemoteController) and a query in the database. It also has to use an high security protocol so we can say that the complexity of this function is \textbf{Average}
\end{itemize}

Summing up all the function spotted we derive the following table:
\begin{center}
\begin{tabular}{|l l l|}
\hline
\multicolumn{1}{|c}{EI}&Complexity&FPs\\
\hline
login/logout & Low & 2x3\\
Registration & Average & 4\\
Reservation Request &Average&4\\
Cancel Reservation & Average&4\\
Report Damaged Car & Average&6\\
Start rent& High & 6\\
End rent & High & 6\\
Pin Insertion &  Average&4\\\hline
\multicolumn{2}{|c}{Total}&40\\
\hline
\end{tabular}
\end{center}

\subsubsection{External Outputs(EO)}
The PowerEnJoy system communicate tos the external world throw several streams and for several purposes, here below we have identified the most important:
\begin{itemize}
\item \textbf{Money saving option:} When the client selects the money saving option the system has to elaborate what is the best charging station based on the destination  and the cars distribution. Once that the computation is done the data is sent to the CarRemoteController and so to the client throw the car display. The computation for this function is algorithmic hard and requires the cooperation of several components so we can classify its complexity as \textbf{Hard}.
\item \textbf{Report damaged/dirty car:} As specified in the RASD document this function allows the user to report to the operator system issues concerning the car from the car display. To accomplish this function are necessary a couple of components and a interaction with an external interface so we can estimate the complexity of the function as \textbf{Average}.
\item \textbf{Signal flat cars:} If the car system finds out that the battery is empty it starts the procedure of signaling the issue to the operator system. This function requires the cooperation of the CarRemoteController, of the CarController and the interaction with an external interface so we can estimate its complexity as \textbf{Average}.
\item \textbf{Notify reservation time expired:} This is a simple operation that requires  only few components to be performed, so we can estimate its complexity as \textbf{Low}.
\item \textbf{Set car status:} This operation is simple  because it requires the cooperation of only two components(CarController, CarCemoteController). We can say that its complexity is \textbf{Low}.
\item \textbf{Display money charged:} For this function are necessary only a chronometer, a multiplication ad to display on the screen a number. All these operation are trivial and made by just one component so we can classify the function as a \textbf{Low} complexity one.
\item \textbf{Send email with pin and password:} After that a client registers in the system it will receive an email with his pin and his password. To accomplish this function are necessary a password and pin generator and an automatic email generator. These are trivial tasks and we can say that the operation has a \textbf{Low} complexity.
\end{itemize} 
\newpage
The overall estimation for external outputs is summarized in the following table: 
\begin{center}
\begin{tabular}{|l l l|}
\hline
\multicolumn{1}{|c}{EO}&Complexity&FPs\\
\hline
Money saving option &High & 7\\
Report damaged/dirty car & Average & 5\\
Signal flat cars &Average&5\\
Notify reservation time expired & Low&4\\
Set car status & Low&4\\
Display money charged& Low & 4\\
Send email with pin and password &  Low&4\\\hline
\multicolumn{2}{|c}{Total}&33\\
\hline
\end{tabular}
\end{center}

\subsubsection{External Inquiries (EQ)}
External inquiries are essential function to retrieve data for our clients. The main inquiries are the one performed in order to retrieve map information and the one to retrieve personal user data. In this paragraph will be summarized the most important:
\begin{itemize}
\item \textbf{Retrieving charging station information:} This function is used by the client in order to populate the map. It requires just few information from the database and are required few components to accomplish this task so we estimate the  inquiry as a \textbf{Low} complexity.
\item \textbf{Car information Retrieval:} This function requires the interaction of few components but the data to be exchanged are a few. It also may require to refresh of the car information making a remote call on the CarRemoteController. This lets us say that the complexity is \textbf{Average}.
\item \textbf{Safe area position:} This function is used by the client app or web app in order to populate the map. It requires a simple query to be accomplished so the complexity estimation for this function is \textbf{Low}.
\item \textbf{Retrieving User Information:} This function is used in the personal area of the client in the web app and the app. Its complexity is \textbf{Low} because it requires just a simple query to  the database in order to be accomplished.
\end{itemize}
In the following table we summarize the estimation of external inquiries: 

\begin{center}
\begin{tabular}{|l l l|}
\hline
\multicolumn{1}{|c}{EQ}&Complexity&FPs\\
\hline
Retrieving charging station information &Low &3\\
Car information Retrieval & Average & 4\\
Safe area position &Low&3\\
Retrieving User Information &  Low&3\\\hline
\multicolumn{2}{|c}{Total}&13\\
\hline
\end{tabular}
\end{center}
\subsubsection{Overall estimation}
The following table summarizes the results of our estimation :
\begin{center}
\begin{tabular}{|l l|}
\hline
Function type&FPs\\
\hline
Internal Logic Files &75\\
External Interface Files & 20\\
External Inputs&40\\
External Outputs&33\\
External Inquiry&12\\\hline
Total&180\\
\hline
\end{tabular}
\end{center}
Considering Java Enterprise Edition as development platform and disregarding the aspects concerning the implementation of the mobile applications (which can be thought as pure presentation with no business logic), we can estimate the total number of lines of code.
Depending on the conversion rates, we have a lower bound of:
\begin{center}
$SLOC = 180 \cdot 46 = 8280$
\end{center}
and an upper bound of :
\begin{center}
$SLOC = 180 \cdot 67 = 12060$
\end{center}
\subsection{Cost and effort estimation:COCOMO II}
In this section we are going to use COCOMO II to provide a cost and effort estimation of the PowerEnJoy project.
COCOMO II (COnstructive COst MOdel) is a complex estimation
model that permits to estimate the number of Person/Month required to develop a complex project using an\textit{ Effort Equation} derived from fitting a regression formula using data from historical projects.
We are going to use the \textit{post-architecture} approach because we have already defined the architecture previously in the Design Document.

\subsubsection{Scale Drivers}
In order to compute the Effort Equation of the COCOMO II model we have to consider the scale drivers of our project. Below is provided the official COCOMO II scale driver table. The selection of scale drivers is based on the rationale that they are a significant source of exponential variation on a project's effort. Each driver has a range of rating levels, from Very Low to Extra High and each rating level has it's own scale factor.

\begin{scaledriverstable}{Scale Factor values for COCOMO II Models}
	Scale Factors & Very Low & Low & Nominal & High & Very High & Extra High\\\hline
	\addfactor{PREC}{thoroughly unprecedented}{largely unprecedented}{somewhat unprecedented}{generally familiar}{largely familiar}{thoroughly familiar}
	\addfactorvalues{6.20}{4.96}{3.72}{2.48}{1.24}{0.00}
	\addfactor{FLEX}{rigorous}{occasional relaxation}{some relaxation}{general conformity}{some conformity}{general goals}
	\addfactorvalues{5.07}{4.05}{3.04}{2.03}{1.01}{0.00}
	\addfactor{RESL}{little (20\%)}{some (40\%)}{often (60\%)}{generally (75\%)}{mostly (90\%)}{full (100\%)}
	\addfactorvalues{7.07}{5.65}{4.24}{2.83}{1.41}{0.00}
	\addfactor{TEAM}{very difficult interactions}{some difficult interactions}{basically cooperative interactions}{largely cooperative}{highly cooperative}{seamless interactions}
	\addfactorvalues{5.48}{4.38}{3.29}{2.19}{1.10}{0.00}
	\addfactor{PMAT}{Level 1 Lower}{Level 1 Upper}{Level 2}{Level 3}{Level 4}{Level 5}
	\addfactorvalues{7.80}{6.24}{4.68}{3.12}{1.56}{0.00}
\end{scaledriverstable}

In order to obtain the scale factors we have chosen for each scale driver the rating value as explained below.

\begin{itemize}
\item  Precedentedness (\textbf{PREC})\\
This driver reflects the experience that developers have with previously developed projects similar to PowerEnJoy.
Since this is our first time managing such a complex project this value will be considered \textsf{Very Low}.
\item  Development flexibility (\textbf{FLEX})\\
This driver reflects the flexibility in the development with respect to pre-established requirements and external interface specifications. We have to conform to strict specifications on the functionalities and to interact with several external systems that have a well defined interface so this value will be considered \textsf{Low}.
\item Risk resolution (\textbf{RESL})\\
This driver reflects the completeness and goodness of the risk management plan. This value will be considered \textsf{High} because we did a detailed Risk Analysis (Chapter 5).
\item Team cohesion (\textbf{TEAM}) \\
This driver reflects how much the team is cohesive.
We are a very cooperative team and we share the same commitment so this value will be considered \textsf{Very High}.
\item Process maturity (\textbf{PMAT}) \\
This driver reflects the process maturity of the organization. In our case we will consider the \textsf{Nominal} value (CMM Level 2) due to our inexperience.

\end{itemize}

The summary of our evaluation is :
\begin{factorcounttable}{Scale Driver}
	Precedentedness(PREC) & Very Low & 6.20\\
	Development flexibility (FLEX) & Low & 4.05\\
	Risk resolution (RESL) & High & 2.83\\
	Team cohesion (TEAM) & Very high & 1.10\\
	Process maturity (PMAT) & Nominal & 4.68\\\hline
	\fptotal{18.86}
\end{factorcounttable}


\subsubsection{Cost Drivers}
In this section we are going to describe which are the cost drivers that needs to be considered in the effort evaluation. Each driver has a range of rating levels, from Very Low to Extra High. For each driver is provided a table that maps the rating level with the effort multiplier.
%TODO Den: indicate post-architecture
%TODO Den: if necessary provide a table with all the cost drivers

\begin{itemize}
	\item Required Software Reliability \textbf{(RELY)}:
	\begin{itemize}
	\item[] This driver reflects the expected software reliability. A downtime of our service would cause problems to passengers but would not lead to high financial losses. Therefore we will consider this value \textsf{Nominal}.
\end{itemize}
\begin{costdriverstable}{RELY Cost Drivers}
	\costdescriptors{RELY Descriptors}{slightly inconvenience}{easily recoverable losses}{moderate recoverable losses}{high financial loss}{risk to human life}{}\hline
	\ratinglevel{Very low}{Low}{Nominal}{High}{Very High}{Extra High}
	\effortmultipliers{0.82}{0.92}{1.00}{1.10}{1.26}{n/a}
\end{costdriverstable}
\end{itemize}

\begin{itemize}
	\item Database size \textbf{(DATA)}:
	\begin{itemize}
	\item[] This driver attempts to capture the effort required to generate the test data during the development. That's strictly related to the amount of data that will be stored in our system. We expect a large amount of data saved in our database due to the complexity of the service provided therefore this value will be considered \textsf{High}.
\end{itemize}
\begin{costdriverstable}{DATA Cost Drivers}
	\costdescriptors{DATA Descriptors}{}{Testing DB bytes/pgm SLOC $<$ 10}{10$\le$D/P$\le$100}{100$\le$/P$\le$000}{DP $>$ 1000}{}\hline
	\ratinglevel{Very low}{Low}{Nominal}{High}{Very High}{Extra High}
	\effortmultipliers{n/a}{0.90}{1.00}{1.14}{1.28}{n/a}
\end{costdriverstable}
\end{itemize}


\begin{itemize}
	\item Product complexity \textbf{(CPLX)}:
	\begin{itemize}
	\item[] We have chosen the rating scale of this cost driver according to the COCOMO II CPLX rating scale table.
In this table Complexity is divided into five areas: control operations, computational operations, device-dependent operations, data management operations, and user interface management operation. The value that fits most our project is \textsf{Very High}.
\newpage
\end{itemize}
\begin{costdriverstable}{CPLX Cost Driver}
	\ratinglevel{Very low}{Low}{Nominal}{High}{Very High}{Extra High}
	\effortmultipliers{0.73}{0.87}{1.00}{1.17}{1.34}{1.74}	
\end{costdriverstable}
\end{itemize}

\begin{itemize}
	\item Required reusability \textbf{(RUSE)}: 
	\begin{itemize}
	\item[] This cost driver reflects the effort needed to construct components intended for reuse on the current or future projects. As we are going to consider only this project is needed reusability "across the project",therefore this value will be set to \textsf{Nominal}.
	\begin{costdriverstable}{RUSE Cost Driver}
		\costdescriptors{RUSE Descriptors}{}{None}{Across project}{Across program}{Across product line}{Across multiple product lines}\hline
		\ratinglevel{Very low}{Low}{Nominal}{High}{Very High}{Extra High}
		\effortmultipliers{n/a}{0.95}{1.00}{1.07}{1.15}{1.24}		
	\end{costdriverstable}
	\end{itemize}
\end{itemize}

\begin{itemize}
	\item Documentation match to life-cycle needs \textbf{(DOCU)}: 
	\begin{itemize}
	\item[]This cost driver reflects  the suitability of the project's documentation to its life-cycle needs. As we have a detailed documentation that covers large part of the life-cycle needs of the project we are going to consider this value \textsf{Nominal}.
	\begin{costdriverstable}{DOCU Cost Driver}
		\costdescriptors{DOCU Descriptors}{Many life-cycle needs uncovered}{Some life-cycle needs uncovered}{Right-sized to life-cycle needs}{Excessive for life-cycle needs}{Very excessive for life-cycle needs}{}\hline
		\ratinglevel{Very low}{Low}{Nominal}{High}{Very High}{Extra High}
		\effortmultipliers{0.81}{0.91}{1.00}{1.11}{1.23}{n/a}		
	\end{costdriverstable}
	\end{itemize}
\end{itemize}

\begin{itemize}
	\item Execution time constraint \textbf{(TIME)}: 
	\begin{itemize}
	\item[] This driver reflects the execution time constraint imposed upon the system. Every rating express the percentage of available execution time that is going to be used by the system. In our case we expect a \textsf{Very High} usage of the execution time because our system is large and complex.
	\begin{costdriverstable}{TIME Cost Driver}
		\costdescriptors{TIME Descriptors}{}{}{$\le$ 50\% use of available execution time}{70\% use of available execution time}{85\% use of available execution time} {95\% use of available execution time}\hline
		\ratinglevel{Very low}{Low}{Nominal}{High}{Very High}{Extra High}
		\effortmultipliers{n/a}{n/a}{1.00}{1.11}{1.29}{1.63}	
	\end{costdriverstable}
	\end{itemize}
\end{itemize}
\newpage
\begin{itemize}
	\item Main storage constraint \textbf{(STOR)}: 
	\begin{itemize}
	\item[] This driver measures the degree of main storage constraint imposed on the system. In the market are available main storage disks large enough for our system. Therefore this value will be considered \textsf{Nominal}.
	\begin{costdriverstable}{STOR Cost Driver}
		\costdescriptors{STOR Descriptors}{}{}{$\le$  50\% use of available storage}{70\% use of available storage}{85\% use of available storage} {95\% use of available storage}\hline
		\ratinglevel{Very low}{Low}{Nominal}{High}{Very High}{Extra High}
		\effortmultipliers{n/a}{n/a}{1.00}{1.05}{1.17}{1.46}	
	\end{costdriverstable}
	\end{itemize}
\end{itemize}

\begin{itemize}
	\item Platform Volatility \textbf{(PVOL)}: 
	\begin{itemize}
	\item[]This driver reflects how frequently a major change of the platform is required. The "platform" is the complex of hardware and software required by the system to perform the tasks. We don't expect frequent changes in our main platform but we forecast to release every six months updates of the client application in order to follow smartphones operatives system progression. According to the table below we will consider this value \textsf{Nominal}.
	\begin{costdriverstable}{PVOL Cost Driver}
		\costdescriptors{PVOL Descriptors}{}{Major change every 12 mo., minor change every 1 mo.}{Major: 6mo; minor: 2wk.}{Major: 2mo, minor: 1wk}	{Major: 2wk; minor: 2 days}{}\hline
		\ratinglevel{Very low}{Low}{Nominal}{High}{Very High}{Extra High}
		\effortmultipliers{n/a}{0.87}{1.00}{1.15}{1.30}{n/a}
	\end{costdriverstable}
	\end{itemize}
\end{itemize}

\begin{itemize}
	\item Analyst Capability \textbf{(ACAP)}: 
	\begin{itemize}
	\item[] The analysis and design of this project have been done thoroughly and precisely. Therefore this value should be set to \textsf{High}.
	\begin{costdriverstable}{ACAP Cost Driver}
		\costdescriptors{ACAP Descriptors}{15th percentile}{35th percentile}{55th percentile}{75th percentile}{90th percentile}{}\hline
		\ratinglevel{Very low}{Low}{Nominal}{High}{Very High}{Extra High}
		\effortmultipliers{1.42}{1.19}{1.00}{0.85}{0.71}{n/a}	
	\end{costdriverstable}
	\end{itemize}
\end{itemize}

\begin{itemize}
	\item Programmer Capability \textbf{(PCAP)}: 
	\begin{itemize}
	\item[] This driver cost should consider the capability of the programmers as a team rather than as individuals. Even if we haven't implemented the project, we value the ability to communicate and cooperate of our team. Therefore we will consider this value as \textsf{Nominal}.
	\begin{costdriverstable}{PCAP Cost Driver}
		\costdescriptors{PCAP Descriptors}{15th percentile}{35th percentile}{55th percentile}{75th percentile}{90th percentile}{}\hline
		\ratinglevel{Very low}{Low}{Nominal}{High}{Very High}{Extra High}
		\effortmultipliers{1.34}{1.15}{1.00}{0.88}{0.76}{n/a}	
	\end{costdriverstable}
	\end{itemize}
\end{itemize}

\begin{itemize}
	\item Application Experience \textbf{(APEX)}: 
	\begin{itemize}
	\item[] The rating of this driver is chose in terms of the team's level of experience with this type of software application. Our experience related to this type of application is very low so this value will be considered \textsf{Very Low}.
	\begin{costdriverstable}{APEX Cost Driver}
		\costdescriptors{APEX Descriptors}{$\le$ 2 months}{6 months}{1 year}{3 years}{6 years}{}\hline
		\ratinglevel{Very low}{Low}{Nominal}{High}{Very High}{Extra High}
		\effortmultipliers{1.22}{1.10}{1.00}{0.88}{0.81}{n/a}
	\end{costdriverstable}
	\end{itemize}
\end{itemize}

\begin{itemize}
	\item Platform Experience \textbf{(PLEX)}: 
	\begin{itemize}
	\item[] Our experience with platforms as graphic user interface, database, networking and Client/Server architecture consist of 1 year. According to the table below we will consider this value \textsf{Nominal}.
	\begin{costdriverstable}{PLEX Cost Driver}
		\costdescriptors{PLEX Descriptors}{$\le$ 2 months}{6 months}{1 year}{3 years}{6 years}{}\hline
		\ratinglevel{Very low}{Low}{Nominal}{High}{Very High}{Extra High}
		\effortmultipliers{1.19}{1.09}{1.00}{0.91}{0.85}{n/a}
	\end{costdriverstable}
	\end{itemize}
\end{itemize}

\begin{itemize}
	\item Language and Tool Experience \textbf{(LTEX)}: 
	\begin{itemize}
	\item[] This driver measures the level of programming language and software tool experience of the project team. Our average experience with Java is almost 1 year, we also have some background on HTML but we don't have any practical experience with JEE. This value will be considered \textsf{Low}.
	\begin{costdriverstable}{LTEX Cost Driver}
		\costdescriptors{LTEX Descriptors}{$\le$ 2 months}{6 months}{1 year}{3 years}{6 years}{}\hline
		\ratinglevel{Very low}{Low}{Nominal}{High}{Very High}{Extra High}
		\effortmultipliers{1.20}{1.09}{1.00}{0.91}{0.84}{n/a}
	\end{costdriverstable}
	\end{itemize}
\end{itemize}

\begin{itemize}
	\item Personnel continuity \textbf{(PCON)}: 
	\begin{itemize}
	\item[] This driver reflects the project's annual personnel turnover. Even if we are not going to develop the project we can expect a low turnover because we suppose to work together toward the project completion. This value will be considered \textsf{Very High}
	\begin{costdriverstable}{PCON Cost Driver}
		\costdescriptors{PCON Descriptors}{48\% / year}{24\% / year}{12\% / year}{6\% / year}{3\% / year}{}\hline	
		\ratinglevel{Very low}{Low}{Nominal}{High}{Very High}{Extra High}
		\effortmultipliers{1.29}{1.12}{1.00}{0.90}{0.81}{n/a}
	\end{costdriverstable}
	\end{itemize}
\end{itemize}
\newpage
\begin{itemize}
	\item Usage of Software Tools \textbf{(TOOL)}: 
	\begin{itemize}
	\item[] We are going to use basic commercial tools so this value will be considered \textsf{Nominal}.
	\begin{costdriverstable}{TOOL Cost Driver}
		\costdescriptors{TOOL Descriptors}{edit, code, debug}{simple, frontend, backend CASE, little integration}{basic life-cycle tools, moderately integrated}{strong, mature life-cycle tools, moderately integrated}{strong, mature, proactive life-cycle tools, well integrated with processes, methods, reuse}{}\hline
		\ratinglevel{Very low}{Low}{Nominal}{High}{Very High}{Extra High}
		\effortmultipliers{1.17}{1.09}{1.00}{0.90}{0.78}{n/a}	
	\end{costdriverstable}
	\end{itemize}
\end{itemize}

\begin{itemize}
	\item Multisite development \textbf{(SITE)}: 
	\begin{itemize}
	\item[]This driver reflects two factors : site collocation and communication support. All the members of our team live in the same city. We also use chats, emails and phone calls to communicate. For this reason we are going to consider this value \textsf{Very High}.

	\begin{costdriverstable}{SITE Cost Driver}
		\costdescriptors{SITE Collocation Descriptors}{Intern-ational}{Multi-city and multi-company}{Multi-city or multi-company}{Same city or metro area}{Same building or complex}{Fully collocated}
		\costdescriptors{SITE Communications Descriptors}{Some phone, mail}{Individual phone, fax}{Narrow band email}{Wideband electronic communication}{Wideband elect. comm., occasional video conf.}{Interactive multimedia}\hline
		\ratinglevel{Very low}{Low}{Nominal}{High}{Very High}{Extra High}
		\effortmultipliers{1.22}{1.09}{1.00}{0.93}{0.86}{0.80}		
	\end{costdriverstable}
	
	\end{itemize}
\end{itemize}

\begin{itemize}
	\item Required development schedule \textbf{(SCED)}: 
	\begin{itemize}
	\item[] We don't have any special schedule constraint imposed for the software developing so this value will be considered \textsf{Nominal}.
	\begin{costdriverstable}{SCED Cost Driver}
		\costdescriptors{SCED Descriptors}{75\% of nominal}{85\% of nominal}{100\% of nominal}{130\% of nominal}{160\% of nominal}{}\hline
		\ratinglevel{Very low}{Low}{Nominal}{High}{Very High}{Extra High}
		\effortmultipliers{1.43}{1.14}{1.00}{1.00}{1.00}{n/a}	
	\end{costdriverstable}
	\end{itemize}
\end{itemize}

\newpage
Overall, our results are expressed by the following table:
\begin{factorcounttable}{Cost Driver}
	Required Software Reliability (RELY) & Nominal & 1.00\\
	Database size (DATA) & High & 1.14\\
	Product complexity (CPLX) & Very high & 1.34\\
	Required Reusability (RUSE) & Nominal & 1.00\\
	Documentation match to life-cycle needs (DOCU) & Nominal & 1.00\\
	Execution Time Constraint (TIME) & Very high & 1.29 \\
	Main storage constraint (STOR) & Nominal & 1.00 \\
	Platform volatility (PVOL) & Nominal & 1.00 \\
	Analyst capability (ACAP) & High & 0.85 \\
	Programmer capability (PCAP) & Nominal& 1.00 \\
	Application Experience (APEX) &Very  Low & 1.22 \\
	Platform Experience (PLEX) & Nominal & 1.00 \\
	Language and Tool Experience (LTEX) & Low& 1.09\\
	Personnel continuity (PCON) & Very High & 0.81 \\
	Usage of Software Tools (TOOL) & Nominal & 1.00 \\
	Multisite development (SITE) & Very high & 0.86 \\
	Required development schedule (SCED) & Nominal & 1.00 \\\hline
	\fpproduct{1,5516}
\end{factorcounttable}

\subsubsection{Effort equation}
Now that we have considered both scale drivers and cost drivers we can estimate the number of Person/Month required to develop the project. In order to do so we compute the following Effort Equation : 
\begin{center}
$$ Effort = A * EAF * KSLOC^E $$
\end{center}
$ A = 2.94 $ approximates a productivity constant in PM/KSLOC.\\ $EAF$ is the product of all cost drivers. In our case it is $ EAF = 1.5516 $. \\Using function points estimation we have estimated :\\

lower bound of $ KSLOC $
\begin{center}
$KSLOC  = 8.280$
\end{center}

upper bound of $ KSLOC $
\begin{center}
$KSLOC = 12.060$
\end{center}
The exponent $E$ is derived from the aggregation of the five Scale Drivers. It is calculated as:
\begin{center}
$$ E = B + 0.01 * \sum_{j=1}^5 SF_j $$
\end{center}
Where $B = 0.91 $ is an empirically proven constant in COCOMO II.\\ We obtain that :
\begin{center}
$ E = 0.91 + 0.01 * 18.86 = 0.91 + 0.1886 \cong 1.0986 $.
\end{center}
\newpage
Finally we can compute the effort needed:\\

lower bound of $ Effort $
\begin{center}
$$ Effort = 2.94 * 1.5516 * 8.280^{1.0986} = 46.52~PM $$
\end{center}

upper bound of $ Effort $
\begin{center}
$$ Effort = 2.94 * 1.5516 * 12.060^{1.0986} = 70.32~PM $$
\end{center}

\subsubsection{Schedule estimation}

We can estimate the project duration using the following equation:
\begin{center}
$$ Duration = 3.67 * Effort^F $$
\end{center}
$Effort$ is the previously estimated effort and $F$ is the schedule exponent derived from the Scale Drivers.
We calculate $F$ using the following formula:
$$ F = D + 0. 2 ∗ (E − B) $$
that is equivalent to write 
$$ F = D + 0.2 * 0.01 * \sum_{j=1}^5 SF_j = 0.28 + 0.2 * 0.01 * 18.86 \cong 0.3177 $$
Now we can calculate:\\

lower bound of $ Duration $
\begin{center}
$$ Duration = 3.67 * 46.52^F = 3.67 * 46.52^{0.3177} \cong 12.43~Months $$
\end{center}

upper bound of $ Duration $
\begin{center}
$$ Duration = 3.67 * 70.32^F = 3.67 * 70.32^{0.3177} \cong 14.17~Months $$
\end{center}

