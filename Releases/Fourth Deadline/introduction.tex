\subsection{Revision History}





\subsection{Purpose and scope}
This  is the \textbf{Project Plan Document} for the \textbf{PowerEnJoy} system. The purpose of this document is to estimate the size, the  complexity and the cost of the project, to identify what are the critical and most difficult tasks, to identify some possible risks proposing some solutions and to arrange an approximate schedule for the project team work.
In the following section we will  first use the function points approach in order to estimate the size of project; identifying the biggest and most crucial function that the system has to provide, highlighting 
what are the Internal and External logic Files (ILF and ELF), the external inputs,outputs and inquiries (EI, EO and EQ) and what are their function points. Then we will estimate the cost of the project in terms of lines of code and person/month using the Cocomo II approach. 
The document continue with the schedule of the  spotted tasks and with the allocation of our resources on the different tasks. 
In the last part  we  will  focus our attention on some of the possible risks to be prevented and we will explain some possible solutions. 




\subsection{Definitions,Acronyms,Abbreviations}






\subsubsection{Definition}
\textbf{Function Points:}A function point is a "unit of measurement" to express the amount of business functionality an information system (as a product) provides to a user. Function points are used to compute a functional size measurement (FSM) of software. \\
\textbf{Cocomo II:}Constructive Cost Model (COCOMO) is a procedural software cost estimation model developed by Barry W. Boehm. The model parameters are derived from fitting a regression formula using data from historical projects.\\
\textbf{Guest:}
 is an user that is not registered yet to PowerEnJoy.\\
\textbf{Client:}
is an user that has completed successfully the registration procedure.\\
\textbf{Logged client:}
is a client that has logged into PowerEnJoy.\\
\textbf{Blocked client:}
is a client that is not allowed to log in to system because he has some debits with PowerEnJoy.\\
\textbf{User:}
a client or a guest.\\
\textbf{Virtuous  behavior:}
the client has a "Virtuous behavior" if he performs one or more of the following actions: 
\begin{itemize}
\item transport at least other two passengers into the car.
\item leaves the car with less than 50\% of the battery empty.
\item plugs the car into the power grid of a charging station.
\end{itemize}
\emph{\\}
\textbf{Rent:} defines the period of time that starts when the logged client opens the car and ends when the door of the car are locked due to a "end rent" request of the logged client.\\
\textbf{Pin:} a personal sequence of four numbers given by the system after the registration, necessary to start the engine.\\
\textbf{Map Information:}
denotes the following information: 
\begin{itemize}
\item location of available cars.
\item location of charging areas.
\item for each charging area the number of available charging spots.
\item boundaries of safe areas.
\end{itemize}
\emph{\\}
\textbf{Not valid data:}
syntactically incorrect data (e.g. mail not in the format local-part@domain).\\
\textbf{Credentials:}
email and password used to log into PowerEnJoy.\\
\textbf{Valid Credentials:}
email and password related to a registered user.\\
\textbf{Car information Menu:}
displayed when a client selects a car on the map. In this menu are displayed the information about the car (remaining charge percentage and license plate number) and a button that can be used to reserve the car.\\
\textbf{Safe Areas:} 
areas in which is permitted to the client to end the rent and leave the car.\\
\textbf{Valid Payment Information:}Credit or prepaid card with at least a minimum amount of money.\\



\subsubsection{Acronyms}
\begin{itemize}
\item \textbf{RASD:} requirements analysis and specifications document
\item \textbf{DD:} design document
\item \textbf{ILF:} Internal Logic Files
\item \textbf{EIF:} External Interface Files
\item \textbf{EI:}External Inputs
\item \textbf{EO:} External Outputs
\item \textbf{EQ:}External Inquiries
\item \textbf{JEE:}Java enterprise edition

\end{itemize}	




\subsubsection{Abbreviations}






\subsection{Reference Documents}

\begin{itemize}
\item RASD Document
\item Design Document
\item Assignment AA 2016-2017.pdf
\item Project planning example document.pdf
\end{itemize}








