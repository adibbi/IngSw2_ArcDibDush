In this section we are going to explain which are the risks that could affect the realization of our project.
Threats could derive from technical issues, from a wrong business analysis or could be related to controversies with our stakeholders.
For each risk we are going to estimate the probability of occurrence and we will provide a possible contingency plan.\\
The first risk that could jeopardize the project comes from aggressive competition. The competitors that provide a car-sharing service are few but consolidated, therefore they could try to initiating a price war in order to discourage new entrants. The probability that this occurs is high and, without the right precautions, this would affect enormously our business. This risk can be mitigated with a good marketing strategy, in particular is necessary to promote our competitive advantage (electric cars and eco-friendly behaviors) in order to differentiate from competitors.\\
Another potential issue could come from the lack of free space where to build the charging stations or from social movements that protest against the realization of charging stations in cultural areas. In both the cases to overcome the problem we have to make clear to the city administration the advantages that PowerEnJoy provides to the local community and to let the stakeholders have an active role in the development of the project through periodical meetings and demonstrations.\\
Other issues that could be critical have an internal origin : key members of the development team could be ill in periods relevant for the project realization or they could just quit the company. This could lead to delays and a lack of knowledge related to a part of the project. To avoid so it's important to maintain a high quality documentation and to organize the team work assuring that duties and responsibilities are assigned across multiple team members so no critical task is assigned to a single person.\\
We have also to consider the possibility that we underestimate the development time necessary to completely develop the project. In this case allocating new budged in order to add manpower to the project is not recommended, if done in a late phase of development would increase even more the delay! Instead is fundamental to promptly alert the customer of the possible delay and possibly accord on a two-phase release in order to provide as soon as possible a working system and later to add and refine functionalities.\\
Other risks might arise from technical issues that could threat the quality of the product and increase the costs of the software developing process.
The application might have security issues if not well designed. This could lead to a leak of users' sensible data that would be catastrophic for PowerEnJoy's image and therefore for the business of the company. In this case we have to follow all the security standards and test very well the most critical parts to lower as much as possible the probability of occurrence.\\
Other risks that should be considered are related to our dependency on external components. The first suppliers of PowerEnJoy are electric cars vendors. The cars must have a centralized system to provide the possibility to lock/unlock doors remotely. Vendors could try to set inconvenient pricing plans for this additional feature. In this case, since PowerEnJoy needs a large amount of cars, we should leverage on the economy of scale to have lower prices.
Other components that have to be acquired externally are the car displays but this shouldn't be an issue due to the large presence of similar devices in the market.

